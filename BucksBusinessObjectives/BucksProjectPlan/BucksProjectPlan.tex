\documentclass[a4paper]{article}
\usepackage{graphicx} 
\usepackage[margin=0.5in]{geometry}
\usepackage{amsmath, booktabs, float, hyperref, graphicx}

\title{Milwaukee Bucks Business Objectives}
\author{Salvin Chowdhury \& Mathias Galvan}
\date{\today}

\begin{document}

\maketitle 

\newpage

\section{Introduction}
In this paper, we understand the customer's business goals and define success from a business perspective. We assess 
the resources, project requirements, risks and perform a cost-benefit analysis. We then define the technical success 
criteria for the data mining work. Lastly, we create a project plan by selecting the right tools and outline the steps.

\section{The Business Objective}
First, we need to understand what we need to understand what the customer (Milwaukee Bucks) wants to achieve with the 
project. This means figuring out their goals or what success looks like for them from a business point of view.

\subsection{The Milwaukee Bucks Business Objectives}
The Milwaukee Bucks are considering the introduction of four themed partial ticket plans for the upcoming season:
\begin{itemize}
    \item \textbf{Value Plan:} focuses on affordable tickets for weekday games
    \item \textbf{Marquee Opponent Plan:} featuring games against high profile opponents
    \item \textbf{Weekend Plan:} hihglighitng weekend games for fans looking for weekend entertainment
    \item \textbf{Promotional Giveaway Inclusive Plan:} centered around games with promotional giveaways
\end{itemize}

\noindent \textbf{Data Science Goal:} to leverage historical ticketing data to predict the likelihood that an account 
will purchase one of the new partial plans and which plan they're most likely to purchase. \\

\noindent \textbf{Business Goal:} to increase sales of new partial ticket plans by targeting fans with personalized 
marketing, based on insights from historical ticketing behavior.

\subsection{Understanding the Data}
To leverage ticketing data means to look at all three data sets and figure out what kind of insights we can derive 
from it. Here are some of the things that we know about the datasets:
\begin{itemize}
    \item There are three kinds of tickets being purchased, single game, partial plan and group tickets
    \item Each user account has a average amount of spending that is made based off the sales of the tickets
    \item There are fans who have a level of interest, have attended games, and live a distance away from the stadium
    \item There are games which contains just data of which games have given out a giveaway
    \item There are games which contains the season at which it was played, as well as the date and the tier
\end{itemize}

\subsection{Cost-Benefit Analysis}
When it comes to availability to ticketing data, we have all the data on the seats, games and the accounts. We can use
data science and machine learning tools to derive insights and deliver conclusions. The timeline for this project is
approximately one month from April to May. \\

\noindent Some associated risks of this may be that the predictions are not accurate or actionable. Another risk is 
that there might not be enough data or there are fields that are missing. Although the cost of this project is that 
time put in for data analysis, the benefit is that this will lead to a more targeted advertising campaign.

\subsection{Data Mining Goals}
When it comes to data mining goals, we would like to predict likelihood of purchase and predict which plan each 
account is likely to purchase. We identify the key customer features in terms of predictiveness, and export these 
predictions into a format that marketing / sales teams can use.

\subsubsection{Data Mining Techniques}
For now, these are some of the suggested data mining techniques that can be used for data analysis:
\begin{itemize}
    \item Determining methods of imputation for the missing values in all of the datasets
    \item Determining the target variable in different datasets and creating visualizations to determine relationships
    \item Using statistical testing tehcniques to determine the relationships between diffent data types
    \item Using hypothesis testing methods to determine the relationships between different features in the dataset
    \item Exploring merging datasets or dividing datasets to derive newer insights of relationships between features
\end{itemize}


\end{document}